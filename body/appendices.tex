\newpage
\appendix

%%附录第一个章节
\section{第一附录}


%%变量列举

\begin{table}[H]
\caption{Symbol Table-Constants}
\centering
\begin{tabular}{lll}
\toprule
Symbol & Definition  & Units\\
\midrule[2pt]
\multicolumn{3}{c}{\textbf{Constants} }\\
$DL$&Expectancy of poisson-distribution &  unitless \\
$NCL$ &Never- Change-Lane& unitless\\
$CCL$&Cooperative-Change-Lane& unitless\\
$ACL$&Aggressive-Change-Lane& unitless\\
$FCL$&Friendly-Change-Lane& unitless\\
$SCC$&Self-driving-Cooperative-Car& unitless\\
$NSC$&None-Self-drive-Car& unitless\\
\bottomrule
\end{tabular}
\end{table}


\section{第二附录}
\textcolor[rgb]{0.98,0.00,0.00}{\textbf{Code}}
\begin{lstlisting}[language=matlab]
% Auto-fit for Monto_Caolo method.
% author: Yin
% time: 2019/12/30 21:11

clear,clc
fun=@(x) exp(x);
n=10000;%选取点的个数
m=0;
%定义积分式的有关运算
a=0;%积分下限
b=1;%积分上限
actual_value=fun(b)-fun(a);
x_temp=linspace(a,b,100);
y_max=1.5*max(abs(fun(x_temp))); %生成因变量绝对值的一个下界
data=zeros(50,2); %用来绘制误差曲线
for k=1:n
x=unifrnd(a,b);%从[a,b]内服从均匀分布随机产生一个数
y=unifrnd(-y_max,y_max);% 从[-y_max,y_max]内服从均匀分布随机产生一个数
if 0<y && y<fun(x)
m=m+1;
elseif y<0 && y>fun(x)
m=m-1;
end
end
p=m/n;
area=p*(2*y_max)*(b-a);
true_value= integral(fun,a,b);
error = abs(true_value-area);
data(1,:)=[n,error];
k_times=0; %在以下while中循环的次数
while error>= 5*10^-3
k_times=k_times+1;
for h=1:10
for i=1:1000
x=unifrnd(a,b);%从[a,b]内服从均匀分布随机产生一个数
y=unifrnd(-y_max,y_max);% 从[-y_max,y_max]内服从均匀分布随机产生一个数
if 0<y && y<fun(x)
m=m+1;
elseif y<0 && y>fun(x)
m=m-1;
end
end
p=m/(n+(k_times-1)*10000+h*1000);
area=p*(2*y_max)*(b-a);
data(k_times+h,:) = [n+(k_times-1)*10000+h*1000,abs(true_value-area)];
end
error=mean(data(k_times+1:k_times+10,2));
end
times=n+10000*k_times;
fprintf('使用Monte-Carlo方法模拟产生散点%d个,计算的积分结果为\n ',times);
fprintf('积分值:%f\n',area);
fprintf('误差:%f\n',error);

plot(data(:,1),data(:,2))
grid on
title('误差曲线');
xlabel('采样点n的个数');
ylabel('误差(数值)');
\end{lstlisting}

\section{第三附录}
\textcolor[rgb]{0.98,0.00,0.00}{\textbf{Code}}
\begin{python}
#!/usr/bin/python
# -*- coding: UTF-8 -*-

if __name__ == '__main__':
i = 10
j = 20
if i > j:
print '%d 大于 %d' % (i,j)
elif i == j:
print '%d 等于 %d' % (i,j)
elif i < j:
print '%d 小于 %d' % (i,j)
else:
print '未知'	
\end{python}

\section{问题4未知日期的求解结果表}
% Table generated by Excel2LaTeX from sheet '视频 day 遍历数据'
\begin{longtable}{ccccc}
	
	\hline
	\hline
	\textbf{天数} & \textbf{杆长(m)} & \textbf{纬度(°)} & \textbf{经度(°)} & \textbf{方差}  \\
	\hline
	\hline
	\endfirsthead
	
	\multicolumn{5}{l}{(接上页)}\\
	
	\hline
	\hline
	\textbf{天数} & \textbf{杆长(m)} & \textbf{纬度(°)} & \textbf{经度(°)} & \textbf{方差}  \\
	\hline
	\hline
	\endhead
	\hline
	\multicolumn{5}{l}{(接下页)}
	
	\endfoot
	\hline
	\hline
	\endlastfoot
	171   & 2     & 39.97  & 109.63  & 5.67E-04 \\
	172   & 2     & 39.96  & 109.63  & 5.67E-04 \\
	170   & 2     & 39.96  & 109.63  & 5.67E-04 \\
	173   & 2     & 39.96  & 109.63  & 5.67E-04 \\
	169   & 2     & 39.96  & 109.64  & 5.67E-04 \\
	174   & 2     & 39.96  & 109.65  & 5.68E-04 \\
	168   & 2     & 39.96  & 109.65  & 5.68E-04 \\
	175   & 2     & 39.96  & 109.66  & 5.68E-04 \\
	167   & 2     & 39.95  & 109.67  & 5.68E-04 \\
	176   & 2     & 39.95  & 109.68  & 5.68E-04 \\
	166   & 2     & 39.95  & 109.70  & 5.68E-04 \\
	177   & 2     & 39.95  & 109.71  & 5.68E-04 \\
	165   & 2     & 39.94  & 109.73  & 5.68E-04 \\
	178   & 2     & 39.94  & 109.74  & 5.69E-04 \\
	164   & 2     & 39.93  & 109.76  & 5.69E-04 \\
	179   & 2     & 39.93  & 109.78  & 5.69E-04 \\
	163   & 2     & 39.93  & 109.80  & 5.69E-04 \\
	180   & 2     & 39.92  & 109.83  & 5.69E-04 \\
	162   & 2     & 39.91  & 109.85  & 5.70E-04 \\
	181   & 2     & 39.91  & 109.87  & 5.70E-04 \\
	161   & 2     & 39.90  & 109.90  & 5.70E-04 \\
	182   & 2     & 39.90  & 109.93  & 5.71E-04 \\
	160   & 2     & 39.89  & 109.96  & 5.71E-04 \\
	183   & 2     & 39.88  & 109.99  & 5.71E-04 \\
	159   & 2     & 39.87  & 110.02  & 5.71E-04 \\
	184   & 2     & 39.87  & 110.05  & 5.72E-04 \\
	158   & 2     & 39.86  & 110.08  & 5.72E-04 \\
	185   & 2     & 39.85  & 110.12  & 5.73E-04 \\
	157   & 2     & 39.84  & 110.15  & 5.73E-04 \\
	186   & 2     & 39.83  & 110.19  & 5.73E-04 \\
	156   & 2     & 39.82  & 110.23  & 5.74E-04 \\
	187   & 2     & 39.81  & 110.27  & 5.74E-04 \\
	155   & 2     & 39.80  & 110.31  & 5.75E-04 \\
	188   & 2     & 39.79  & 110.35  & 5.75E-04 \\
	154   & 2     & 39.77  & 110.40  & 5.76E-04 \\
	189   & 2     & 39.76  & 110.44  & 5.76E-04 \\
	153   & 2     & 39.75  & 110.49  & 5.77E-04 \\
	190   & 2     & 39.73  & 110.54  & 5.77E-04 \\
	152   & 2     & 39.72  & 110.58  & 5.78E-04 \\
	191   & 2     & 39.70  & 110.63  & 5.78E-04 \\
	151   & 2     & 39.69  & 110.68  & 5.79E-04 \\
	192   & 2     & 39.67  & 110.74  & 5.79E-04 \\
	150   & 2     & 39.66  & 110.79  & 5.80E-04 \\
	193   & 2     & 39.64  & 110.84  & 5.80E-04 \\
	149   & 2     & 39.62  & 110.90  & 5.81E-04 \\
	194   & 2     & 39.60  & 110.96  & 5.82E-04 \\
	148   & 2     & 39.58  & 111.01  & 5.82E-04 \\
	195   & 2     & 39.56  & 111.07  & 5.83E-04 \\
	147   & 2     & 39.54  & 111.13  & 5.84E-04 \\
	196   & 2     & 39.52  & 111.19  & 5.84E-04 \\
	146   & 2     & 39.50  & 111.26  & 5.85E-04 \\
	197   & 2     & 39.48  & 111.32  & 5.86E-04 \\
	145   & 2     & 39.46  & 111.38  & 5.86E-04 \\
	198   & 2     & 39.43  & 111.45  & 5.87E-04 \\
	144   & 2     & 39.41  & 111.52  & 5.88E-04 \\
	199   & 2     & 39.38  & 111.58  & 5.89E-04 \\
	143   & 2     & 39.36  & 111.65  & 5.89E-04 \\
	200   & 2     & 39.33  & 111.72  & 5.90E-04 \\
	142   & 2     & 39.30  & 111.79  & 5.91E-04 \\
	201   & 2     & 39.27  & 111.86  & 5.92E-04 \\
	141   & 2     & 39.24  & 111.94  & 5.93E-04 
	\label{tab:apptab1}%
	
\end{longtable}%